\documentclass[a4paper,10pt]{article}
\usepackage[utf8]{inputenc}

%opening
\title{Report: Practical 2 \\ Genetic algorithms}
\author{Michael Glazunov, Martin Gaßner}
\date{\today}

\begin{document}
%title page
\pagenumbering{gobble}
\maketitle

\newpage
\pagenumbering{arabic}

\section{Problem definintion}
In the following document, we are going to describe a genetic algorithm used to generate a desired sequence of characters. We will outline our choices and experiments and the conclusions we draw therefrom.\newline
Since the model of an individual is given and we did not change its original implementation, we will avoid further discussion.

\section{Approaches to solve the problem}
\subsection{Modelling}
We decided to split the program into seperate tasks, modelling seperate activities of a genetic algorithm. These are:
\begin{enumerate}
 \item Recombinators model the process of generating a new individual as a combination of the genes of two parents.
 \item Mutators model the process of mutation a new individual may undergo.
 \item Evaluators model different fitness functions.
 \item Selectors model the evolutionary selection and thus exert pressure on the population.
 \item Terminators indicate when the population has evolved enough.
\end{enumerate}
\subsection{Implementation of models}
Because the above mentionned models are very abstract we now define the concrete implementation of thse concepts.

\subsubsection{Recombinators}
\begin{enumerate}
 \item Crossover recombination generates a new genome from a subsequence of the genome of both parents. These subsequences are then attached to form the new genome.
 \item Alternating recombination generates a new genome by alternatingly taking one gene from one of the parents, adding it to the end of the newly forming genome.
\end{enumerate}

\subsubsection{Mutators}
\begin{enumerate}
 \item Uniform mutation treats all individuals the same, thus mutating one gene of every individual with the same probability using a uniform distribution.
 \item Fitness dependent mutation is more likely to mutate a gene of an individual with low fitness. The fitness value as given by an evaluator is directly proportional to the probability of mutation.
\end{enumerate}

\subsubsection{Evaluators}
\begin{enumerate}
 \item Edit distance is an adapted version of the actual edit distance: It computes the ratio of the number of correct genes over the number of incorrect genes.
\end{enumerate}

\subsubsection{Selectors}
\begin{enumerate}
 \item Elitist selection sorts the list of individuals and discards the least performing individuals exceeding the size of the population.
 \item Probabilistic selection selects an individual based on a probability proportional to the individual's fitness value.
    \subitem Uniform probabilistic selection uses a uniform distribution.
    \subitem Normal probabilistic selection uses a normal distribution. For this model individuals performing better than average have a higher chance of surviving as opposed to the previous model, whereas individuals with a lower than average fitness value are even more likely to be discarded.
\end{enumerate}

\subsection{Terminator}
\begin{enumerate}
 \item Finite termination generates a finite number of generation until the evolutionary process is terminated.
 \item Stable solution termination generates as many generations as necessary to produce a stable solution. This means that the algorithm terminates once the best solution does not change anymore over a specified number of generations.
\end{enumerate}






\section{Conclusions}

\end{document}
